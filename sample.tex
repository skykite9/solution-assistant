\documentclass[oldfontcommands]{oblivoir}%nokorean if you are writing in English. 

\usepackage[korean]{solution-math}%default = english 



\tcbset{
.exercise_box/.style={ 
        colback=red!5!white,colframe=red!75!black,}
}%tcolorbox option

\newtheorem{exercise}{Exercise}[.exercise_box]
\newtheorem{problem}{Problem}[.exercise_box]
\newtheorem*{criteria}{Criteria}

\title{Scoring criteria for Homework 4}
\author{Hyunwoo Kwon}
\course{MAT2210, Advanced Calculus II}

 

\begin{document}
\maketitle 

\begin{itemize}
\item 답안의 완성도가 떨어질 경우 감점 1점
\item 고등학교 교육과정에서 배우지 않은 부등식을 사용할 경우(교과서에서, 참고서 아님) 반드시 증명을 제시할 것. 
\item 교과서 본문에 없는 정리를 증명없이 사용할 경우 원칙적으로 점수를 부여하지 않음. 단 과제로 이미 푼 문제는 자유롭게 사용 가능. 
\item 본문의 연습문제를 사용해야 할 경우(유기적으로 연결된 문제의 경우)에는 인용을 할 경우 점수를 인정. 
\item 각 문장 하나하나가 논리적 연결이 이루어지도록 답안을 작성하는 것을 원칙으로. 
\end{itemize}
 
 
 
 
 \begin{exercise}[number=6.29,point=10pt]\label{ex:closed and bounded-max and min}
 If  $E$ is  a nonempty closed and bounded subset of $\mathbb{R}$, then there exist $m , M \in E$ such that $m \le x \le M$ for all $x \in E$.
 \end{exercise} 
 \begin{criteria}\,
 \begin{itemize}
 \item  completeness axiom에 의하여 $\inf E$와 $\sup E$의 존재성을 체크했을 경우 (조건 체크 없이 이와 같은 숫자들이 존재한다고 했으면 감점) (+5)
 \item limit을 이용해서 보일 때, $\sup E$가 $E$의 원소들로 이루어진 수열로 접근한다는 것을 보였을 경우 (+5) / limit point임을 보이고 $E$가 closed임을 이용해서 보여도 됨. 
 \item 또는 모순법을 이용해서 $E^c$가 open임을 이용해서 보일 경우 (+5)
 \end{itemize}
 
 * interior point가 없는 집합도 있음.  
 \end{criteria}
  
  
 
 \begin{problem}[number=6.13,name=Cantor's Nested Theorem,point=10pt]
 A sequence $\{E_k \}$ of sets in $\mathbb{R}^n$ is said to be {\it nested} if $E_{k+1} \subset E_k$ for all $k\in \mathbb{N}$. Prove that if $\{K_m\}$ is a nested sequence  of nonempty compact subsets of $\mathbb{R}^n$, then  $\bigcap_{m=1}^\infty K_m$ is nonempty and compact.
 Prove also that  if ${\rm diam}\, (K_m ) \to 0$ as $m \to \infty$ in addition, then  $\bigcap_{m=1}^\infty K_m$ consists of only one point.
 \end{problem}
 \begin{criteria}\label{test}
 Bolzano-Weierstrass theorem을 이용해서 증명하거나, 결론을 부정해서 compactness의 정의를 이용해서 증명한 경우 7점
 
 ${\rm diam}\, (K_m ) \to 0$ as $m \to \infty$ 가정하에 원하는 것을 증명했을 경우 3점
 
 Cauchy sequence를 잡는 작업을 한 사람도 있는데, 두 번 증명하는 것이라 불 필요. 
  \end{criteria}
   
 



\end{document}
